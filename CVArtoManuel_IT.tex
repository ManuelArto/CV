\documentclass{tccv}
\usepackage[english]{babel}
\usepackage[utf8]{inputenc}
\usepackage[T1]{fontenc}

\usepackage{fontawesome}
\newcommand\SYMBOL[1]{\raisebox{-2pt}{\Large\ding{#1}}}

\begin{document}

\part{Manuel Arto}

\section{About Me}

Appassionato di informatica sin dall'infanzia, sto attualmente svolgendo il corso di Magistrale Informatica presso l'Università di Bologna. Ho concluso la triennale svolgendo contemporanemente un lavoro part-time come sviluppatore backend. I miei interessi principali sono lo sviluppo backend e blockchain.

\section{Esperienze Lavorative}

\begin{eventlist}

\item{Set 2021 - Set 2023}
     {Net Service, Bologna}
     {Software Engineer} \\
     \textbullet~Gestione e sviluppo di una piattaforma per la raccolta di dati relativi a procedimenti giudiziari per oltre 40 Istituti di Credito. \\
     \textbullet~Sviluppo di un sistema a 5 microservizi utilizzando Spring Boot e OpenShift per il deployment. \\
     \textbullet~Migrazione da architettura silos a multitenant permettendo un risparmio di più del 80\% in termini di risorse. \\
     \textbullet~Integrazione di Apache Camel, ActiveMQ e sviluppo di script ETL per gestire un flusso di dati di oltre 100 milioni di record. \\
     \textbullet~Test-Driven Development (TDD) tramite JUnit per lo sviluppo di nuove funzionalità, per un coverage dell'75\%.

\item{Ott 2019}
     {MySponge, Vilnius Lithuania}
     {Erasmus +} \\
     \textbullet~Miglioramenti UI/UX tramite Wordpress del sito di e-commerce. \newline
     \textbullet~Integrato sistema di monitoraggio per visibilità dei prodotti in oltre 100 punti vendita. \newline
     \textbullet~Utilizzo di Trello come tool per task management.

\end{eventlist}


\section{Progetti Personali}

\begin{yearlist}

\item{2023}
     {\href{https://github.com/manuelarto/livechat}{Livechat}}
     {\textbullet~Applicazione mobile con funzionalità social avanzate come real-time chat, condivisione della posizione e una classifica basata sul conteggio dei passi giornalieri. \newline
    \textbullet~ Python, Flutter, Websocket, MongoDB, JWT}
\item{2022}
     {\href{https://github.com/manuelarto/crazyplayer}{CrazyPlayer}}
     {\textbullet~AI Player in grado di giocare in modo ottimale in tutte le istanze possibili del M,N,K-game. \newline
     \textbullet~Classificato al primo posto nel torneo tra i vari player creati dagli studenti  \newline
    \textbullet~Java, Game Tree, Algorithms, Data Structure}
\end{yearlist}


\newpage


\begin{keyvaluelist}{personal}
    \item[\faHome] Bologna
    \item[\faPhone] 388 7833963
    \item[\faEnvelope] \href{mailto:my@email.address}{manuelarto01@gmail.com}
    \item[\faGithub] \href{https://github.com/manuelarto}{Github}
    \item[\faLinkedin] \href{https://www.linkedin.com/in/manuel-arto-696012203/}{Linkedin}
\end{keyvaluelist}


\section{Istruzione}

\begin{yearlist}

\item[Laurea Magistrale]{2023-2025}
     {Informatica}
     {Università di Bologna}

\item[Laurea Triennale]{2020-2023}
    {Informatica}
    {Università di Bologna}

\item[Scuola Superiore]{2015-2020}
    {Tecnico Informatico \newline 100/100}
    {IIS Belluzzi, Bologna}

\end{yearlist}


\section{Corsi}

\begin{yearlist}

\item[Udemy]{Mag 2020}
     {Flutter\&Dart - Complete Guide}
     {42h, \href{https://www.udemy.com/certificate/UC-c6f5a32f-babc-42f9-8a0a-6effadf9e7cd/}{link certificato}}

\item[Youtube]{Ott 2023}
    {Blockchain Developer, Smart Contract, \& Solidity Course}
    {Corso completo da principiante a esperto, \href{https://github.com/Cyfrin/foundry-full-course-f23}{repo github}}

\end{yearlist}


\section{Competizioni}

\begin{yearlist}

\item[]{Mag 2019}
     {Gara Nazionale di Informatica}
     {Classificato 5° in tutta Italia \newline
     Progettazione UML e sviluppo Java}

\item[]{2019}
    {Olimpiadi Italiane Informatica}
    {Competitive programming in C++}

\end{yearlist}


\section{Skills}

\begin{factlist}

\item{Technical}
     {Python, Java, Solidity, Go, Node.js, Flutter, C++, Svelte, Vue, Spring Boot, FastAPI, MongoDB, PostgresSQL, Blockchain, Docker, Kubernetes, ETL, Git, Design, Testing}

\item{Soft}
     {Collaborativo, Communicazione, Rapidità, Passione}

\end{factlist}


\section{Hobby}

Calistenichs, Scacchi, Film, Serie TV, Anime, Manga


\end{document}
